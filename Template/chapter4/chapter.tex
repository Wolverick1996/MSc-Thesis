% !TEX root = ../thesis.tex
%\chapter{[Core contribution]: Approach}
\chapter{Experiments}
\label{capitolo4}
\thispagestyle{empty}

\iffalse

This is the chapter where you explain how you approach the problem and how you intend to meet the requirements identified in the previous chapter. In short, here you explain your \emph{solution}. But attention: you won't be able to describe every aspect of your thesis project here, in one single chapter. You will need more than one for that. So, this is the chapter where you explain your solution in terms of the general approach and the design decisions that you make:

\begin{itemize}
\item[\Square] Identify the target \emph{actors} that will benefit from your solution, describe them.
\end{itemize}

\fi


\section{Our Societal Focus: Gender Gap}
Modern societies, as well as every society in human history, are faced with various and different social problems (i.e. conditions or behaviors that have negative consequences for large numbers of people and that are generally recognized as conditions or behaviors that need to be addressed). Global climate change, overpopulation, poverty and homelessness, or racial discrimination are just a few examples -- the tip of the iceberg -- but many more social problems exist and are far from being resolved. For the purpose of our research, we decided to focus on discrimination problems, and in particular on the so called ``\textbf{gender gap}''. According to \cite{cambridge2013gender}, gender gap is definable as:
\begin{quote}\emph{A difference between the way men and women are treated in society, or between what men and women do and achieve.} \cite{cambridge2013gender}\end{quote}
Specifically, the focus of our experiments is \textit{gender pay gap}, already mentioned in Section~\ref{section:the_glassdoor_method}, that is, the average difference between the remuneration for men and women in the workforce, or, in other words, a measure of what women are paid relative to men. The experiments are centered on the economical perspective because it is the easiest to measure in the data, being quantifiable for example as a number representative of a person's average monthly salary, but other facets of the gender gap problem will come into play when dealing with sociological studies.


\section{Datasets Description}
The main purpose of this research is to combine the technological perspective with the sociological one, in order to analyze the strenghts and weaknesses of the adopted tools in real-world scenarios. For this reason, we decided to use real-world datasets, containing information related to public employees of the U.S., and more specifically of public employees working in the cities of Chicago\footnote{Available at: \url{https://www.chicago.gov/city/en/depts/dhr/dataset/current_employeenamessalariesandpositiontitles.html}.} and San Francisco\footnote{Available at: \url{https://www.kaggle.com/tomtillo/san-francisco-city-payrollsalary-data-20112019}.}.

The \textbf{Chicago} dataset we considered includes 31,858 tuples and is made up of 8 attributes, briefly described as follows:
\begin{itemize}
\item \textit{Name}: full name of the employee in the form of ``Surname, Name''.
\item \textit{Job Titles}: categorical variable representing the job title of the employee (e.g. POLICE OFFICER). There are 1089 distinct values.
\item \textit{Department}: categorical variable representing the job department where the employee works (e.g. POLICE). There are 36 distinct values.
\item \textit{Full or Part-Time}: binary categorical variable describing whether the employee is employed full-time (F) or part-time (P).
\item \textit{Salary or Hourly}: binary categorical variable describing whether the employee is paid on a hourly basis or salary basis. Hourly employees are further defined by the number of hours they work in a week.
\item \textit{Typical Hours}: numerical variable describing the typical amount of work (in terms of number of hours per week) for hourly employees. For salary employees the attribute value is null.
\item \textit{Annual Salary}: numerical variable describing the annual salary rate. It only applies for employees whose pay frequency is Salary, while for hourly employees the attribute value is null.
\item \textit{Hourly Rate}: numerical variable describing hourly salary rates for employees whose pay frequency is Hourly. For salary employees the attribute value is null.
\end{itemize}

The \textbf{San Francisco} dataset we considered includes 357,407 tuples and is made up of 10 attributes, briefly described as follows:
\begin{itemize}
\item \textit{Employee Name}: full name of the employee in the form of ``Name Surname''.
\item \textit{Job Title}: categorical variable representing the job title of the employee (e.g. Firefighter). There are 2306 distinct values.
\item \textit{Base Pay}: numerical variable describing the annual regular pay for the employee.
\item \textit{Overtime Pay}: numerical variable describing the annual overtime pay for the employee.
\item \textit{Other Pay}: numerical variable describing other annual pay components for the employee.
\item \textit{Benefits}: numerical variable describing the amount of annual benefits for the employee.
\item \textit{Total Pay}: numerical variable describing the total annual salary of the employee, benefits excluded (\textit{Base Pay} + \textit{Overtime Pay} + \textit{Other Pay}).
\item \textit{Total Pay + Benefits}: numerical variable describing the total annual salary of the employee, benefits included (\textit{Base Pay} + \textit{Overtime Pay} + \textit{Other Pay} + \textit{Benefits}).
\item \textit{Year}: numerical variable representing the year of reference (the dataset contains data related to the years 2011 to 2019).
\item \textit{Status}: binary categorical variable describing whether the employee is employed full-time (FT) or part-time (PT).
\end{itemize}


\section{Case Study 1: Chicago}
\subsection{Data Preprocessing}
In order to simplify the subsequent bias analysis, we operated some \textbf{data transformation} processes on the attributes, choosing what we believe to be the most suitable names. For the Chicago dataset, we renamed \textit{Job Titles} in \textit{Job Title} and \textit{Full or Part-Time} in \textit{Status}. We also performed some \textbf{data aggregation}, estimating the \textit{Annual Salary} of hourly employees by using the formula \textit{Typical Hours} * \textit{Hourly Rate} * 52, where 52 is a constant representing the number of weeks in a year.

Since our focus is gender pay gap but the original datasets do not contain a \textit{Gender} attribute, we adopted a Python package called \texttt{gender-guesser}\footnote{Available at: \url{https://pypi.org/project/gender-guesser}.}. The aim of the package is to infer a person's gender from their first name, and the possible outcomes are: unknown (name not found), andy (androgynous), male, female, mostly\_male, or mostly\_female. The difference between andy and unknown is that the former is found to have the same probability to be male than to be female, while the latter means that the name was not found in the database. For each employee, we split the \textit{Name} attribute to obtain their \textit{First Name}, and then we inferred their gender by using the package. We obtained (out of the total of 31,858 tuples):
\begin{itemize}
\item unknown: 2653 values.
\item andy: 184 values.
\item male: 20,562 values.
\item female: 6,954 values.
\item mostly\_male: 775 values.
\item mostly\_female: 730 values.
\end{itemize}
In order to get coherent results in case of multiple experiments on the same dataset, we decided to remove the tuples related to unknown and androgynous names instead of randomly assign a gender to them (otherwise, we would have get different numbers at each execution of the preprocessing algorithm). Furthermore, we assumed mostly male names to be effectively related to males and mostly female names to be effectively related to females, and therefore we got 21,337 male values and 7,684 female values for a newly generated \textit{Gender} attribute as a result of this first \textbf{data cleaning} process.

We also operated \textbf{data reduction} by removing the \textit{Typical Hours}, \textit{Hourly Rate} and \textit{First Name} columns, not relevant for our analysis.

As a consequence of the first data cleaning process, the number of different job titles decreased from 1,089 to 1,057. However, since the FAIR-DB tool used for bias analysis requires user interactions, and in order to lighten the workload and speed up computational times, we decided to remove job titles with less than 100 occurrences.

Our final preprocessed dataset includes 20,309 tuples, of which 16,146 males and 4,163 females, and with 35 distinct \textit{Job Title} values.


\subsection{The ``Glassdoor Method''}
As already specified in Section~\ref{section:the_glassdoor_method}, the point of reference for this method is the report published by Glassdoor in 2017 with the aim of helping HR practitioners in analyzing the internal gender pay gap of their companies \cite{chamberlain2017analyze}. Although the report provides a step-by-step guide for the statistical software R, we decided to use Python for our analysis, in order to better integrate the results with the ones from the other tools.

The first step of the analysis, after cleaning up the data and loading them, consists in the creation of a couple of attributes useful for the statistical analysis: \textit{Log Annual Salary} and \textit{Male}. The former is simply the natural logarithm of the annual salary of the employee (i.e. the logarithm to the base of the mathematical constant \textit{e}, approximately equal to 2.71828), useful because it provides a simple interpretation of the regression results; the latter is a dummy indicator equal to 1 for males and 0 for females, which is the key variable of the analysis: if there is no gap, being male should not provide any advantage, and the coefficient of this variable in the regression should be equal to 0, otherwise its value would give us an estimate of the approximate percentage pay gap between men and women. It is worth to mention that the Glassdoor report also suggests to perform a discretization of age values of employees, grouping them into bins (below 25, 25-34, 35-44, 45-54, 55+), but our dataset does not contain any information about the age of the employees.

The report suggests to look at the data before proceeding with the regressions, and it recommends to print a ``summary table'' displaying the basic statistical information about the dataset. Figure~\ref{fig:chicago_glassdoor1}(a) shows the table related to the Chicago dataset, and it displays for the variables \textit{Annual Salary}, \textit{Log Annual Salary} and \textit{Male} sample size (count), arithmetic mean, minimum and maximum values, and standard deviation (i.e. a measure of the average amount of variability in the dataset -- calculated as the square root of the variance -- which tells, on average, how far each value lies from the mean).
Another useful visualization tool is the so called ``pivot table'', displayed in Figure~\ref{fig:chicago_glassdoor1}(b), which provides a high-level summary of the overall difference in pay between men and women by showing the arithmetic mean of the \textit{Annual Salary} attribute values for males and females, together with the number of observations (len) and the median values (i.e. the numeric values separating the higher half of the samples from the lower half). The pivot table is also useful to get a first estimate of the ``unadjusted'' pay gap: men on average are paid \$92,022.03 per year, while women on average earn \$79,790.83 per year -- an overall ``unadjusted'' pay gap of \$12,231.2 (13.3\% of male pay).
Lastly, since we are also interested in the ``adjusted'' pay gap, it is important to look at the average salaries of men and women employed in the different job titles. Figure~\ref{fig:chicago_glassdoor1}(c) shows the first 8 (out of 35) job titles in alphabetical order, displaying average salaries for men and women and sizes of the samples (i.e. number of men and women employed in the specific job title -- information relevant to the problem of representation).

\begin{figure}[h!]
\centering
\noindent\rule{\linewidth}{0.4pt}\par
%\resizebox{\linewidth}{!}{%
\scalebox{.9}{\BVerbatimInput{figures/chicago_glassdoor1a.txt}}
\caption*{(a)}
\noindent\rule{\linewidth}{0.4pt}\par
%\resizebox{\linewidth}{!}{%
\scalebox{.9}{\BVerbatimInput{figures/chicago_glassdoor1b.txt}}
\caption*{(b)}
\noindent\rule{\linewidth}{0.4pt}\par
%\resizebox{\linewidth}{!}{%
\scalebox{.9}{\BVerbatimInput{figures/chicago_glassdoor1c.txt}}
\caption*{(c)}
\noindent\rule{\linewidth}{0.4pt}
\caption{Summary table (a), pivot table (b) and average salaries of men and women employed in the different job titles (c) for the Chicago dataset.}
\label{fig:chicago_glassdoor1}
\end{figure}

In order to estimate the gender pay gap, the reference linear regression model to estimate, as mentioned in Section~\ref{section:the_glassdoor_method}, is: \[\mathit{Log Annual Salary}_i = \beta_1\textit{Male}_i + \beta_2\mathit{Controls}_i + \epsilon_i\]
The report recommends to run three different models: the first with no controls at all, regressing salary only on the male-female gender dummy (and therefore calculating the approximate overall percentage pay gap between men and women -- the ``unadjusted'' pay gap); the second with the addition of variables related to employee characteristics like highest education, years of experience, and performance evaluation scores; the third including all the possible controls (and finally estimating the ``adjusted'' pay gap). Due to the lack of attributes, we performed only two linear regressions: the first with no controls and the second including \textit{Job Title}, \textit{Department}, and \textit{Status}.

\begin{figure}[h!]
\centering
\noindent\rule{\linewidth}{0.4pt}\par
%\resizebox{\linewidth}{!}{%
\scalebox{.9}{\BVerbatimInput{figures/chicago_glassdoor2.txt}}
\noindent\rule{\linewidth}{0.4pt}
\caption{Regression results for the Chicago dataset.}
\label{fig:chicago_glassdoor2}
\end{figure}

The results are shown in Figure~\ref{fig:chicago_glassdoor2}: a coefficient of 0.242 on the male-female dummy variable means there is approximately 24.2\% ``unadjusted'' pay gap (therefore, men on average earn 24.2\% more than women), but adding to the model all of the controls available in the data the coefficient value shrinks to 0.4\% and becomes no longer statistically significant. In this case, we say there's no evidence of a systematic gender pay gap on an ``adjusted'' basis, after controlling for observable differences between male and female workers, and the big discrepancy between the coefficient values is due to the overrepresentation of men in higher-paying roles and their underrepresentation in lower-paying jobs.


\newpage
\subsection{FAIR-DB}
\subsection{Ranking Facts}


\section{Case Study 2: San Francisco}
\subsection{Data Preprocessing}
\subsection{The ``Glassdoor Method''}
\subsection{FAIR-DB}
\subsection{Ranking Facts}


\section{Other Design Choices}


\iffalse

\note{Figures and tables}{You are an engineer, and using figures (illustrations) and tables to better convey your ideas should be an obvious practice you should have learned throughout your university career. If not, it's time now. Use illustrations, screen shots, sketches, and so on to help the reader understand. Use tables to summarize complex text (for example, a profound analysis of the state of the art) or to format data in a readable fashion. Each time you use a figure or table, you must also (i) complement it with a so-called caption (a text right underneath or above it) to give it a title and a description and (ii) reference it from within the main text (never just place a figure somewhere without talking about it). If you use Latex, check your Latex documentation for how to use captions and references.}

\fi
