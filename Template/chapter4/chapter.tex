% !TEX root = ../thesis.tex
%\chapter{[Core contribution]: Approach}
\chapter{Experiments}
\label{capitolo4}
\thispagestyle{empty}

\iffalse

This is the chapter where you explain how you approach the problem and how you intend to meet the requirements identified in the previous chapter. In short, here you explain your \emph{solution}. But attention: you won't be able to describe every aspect of your thesis project here, in one single chapter. You will need more than one for that. So, this is the chapter where you explain your solution in terms of the general approach and the design decisions that you make:

\begin{itemize}
\item[\Square] Identify the target \emph{actors} that will benefit from your solution, describe them.
\end{itemize}

\fi


\section{Our Societal Focus: Gender Gap}
Modern societies, as well as every society in human history, are faced with various and different social problems (i.e. conditions or behaviors that have negative consequences for large numbers of people and that are generally recognized as conditions or behaviors that need to be addressed). Global climate change, overpopulation, poverty and homelessness, or racial discrimination are just a few examples -- the tip of the iceberg -- but many more social problems exist and are far from being resolved. For the purpose of our research, we decided to focus on discrimination problems, and in particular on the so called ``\textbf{gender gap}''. According to \cite{cambridge2013gender}, gender gap is definable as:
\begin{quote}\emph{A difference between the way men and women are treated in society, or between what men and women do and achieve.} \cite{cambridge2013gender}\end{quote}
Specifically, the focus of our experiments is \textit{gender pay gap}, already mentioned in Section~\ref{section:the_glassdoor_method}, that is, the average difference between the remuneration for men and women in the workforce, or, in other words, a measure of what women are paid relative to men. The experiments are centered on the economical perspective because it is the easiest to measure in the data, being quantifiable for example as a number representative of a person's average monthly salary, but other facets of the gender gap problem will come into play when dealing with sociological studies.


\section{Datasets Description}
The main purpose of this research is to combine the technological perspective with the sociological one, in order to analyze the strenghts and weaknesses of the adopted tools in real-world scenarios. For this reason, we decided to use real-world datasets, containing information related to public employees of the U.S., and more specifically of public employees working in the cities of Chicago\footnote{Available at: \url{https://www.chicago.gov/city/en/depts/dhr/dataset/current_employeenamessalariesandpositiontitles.html}.} and San Francisco\footnote{Available at: \url{https://www.kaggle.com/tomtillo/san-francisco-city-payrollsalary-data-20112019}.}.

The \textbf{Chicago} dataset we considered includes 31,858 tuples and is made up of 8 attributes, briefly described as follows:
\begin{itemize}
\item \textit{Name}: full name of the employee in the form of ``Surname, Name''.
\item \textit{Job Titles}: categorical variable representing the job title of the employee (e.g. POLICE OFFICER). There are 1089 distinct values.
\item \textit{Department}: categorical variable representing the job department where the employee works (e.g. POLICE). There are 36 distinct values.
\item \textit{Full or Part-Time}: binary categorical variable describing whether the employee is employed full-time (F) or part-time (P).
\item \textit{Salary or Hourly}: binary categorical variable describing whether the employee is paid on a hourly basis or salary basis. Hourly employees are further defined by the number of hours they work in a week.
\item \textit{Typical Hours}: numerical variable describing the typical amount of work (in terms of number of hours per week) for hourly employees. For salary employees the attribute value is null.
\item \textit{Annual Salary}: numerical variable describing the annual salary rate. It only applies for employees whose pay frequency is Salary, while for hourly employees the attribute value is null.
\item \textit{Hourly Rate}: numerical variable describing hourly salary rates for employees whose pay frequency is Hourly. For salary employees the attribute value is null.
\end{itemize}

The \textbf{San Francisco} dataset we considered includes 357,407 tuples and is made up of 10 attributes, briefly described as follows:
\begin{itemize}
\item \textit{Employee Name}: full name of the employee in the form of ``Name Surname''.
\item \textit{Job Title}: categorical variable representing the job title of the employee (e.g. Firefighter). There are 2306 distinct values.
\item \textit{Base Pay}: numerical variable describing the annual regular pay for the employee.
\item \textit{Overtime Pay}: numerical variable describing the annual overtime pay for the employee.
\item \textit{Other Pay}: numerical variable describing other annual pay components for the employee.
\item \textit{Benefits}: numerical variable describing the amount of annual benefits for the employee.
\item \textit{Total Pay}: numerical variable describing the total annual salary of the employee, benefits excluded (\textit{Base Pay} + \textit{Overtime Pay} + \textit{Other Pay}).
\item \textit{Total Pay + Benefits}: numerical variable describing the total annual salary of the employee, benefits included (\textit{Base Pay} + \textit{Overtime Pay} + \textit{Other Pay} + \textit{Benefits}).
\item \textit{Year}: numerical variable representing the year of reference (the dataset contains data related to the years 2011 to 2019).
\item \textit{Status}: binary categorical variable describing whether the employee is employed full-time (FT) or part-time (PT).
\end{itemize}


\section{Case Study 1: Chicago}
\subsection{Data Preprocessing}
In order to simplify the subsequent bias analysis, we operated some \textbf{data transformation} processes on the attributes, choosing what we believe to be the most suitable names. For the Chicago dataset, we renamed \textit{Job Titles} in \textit{Job Title} and \textit{Full or Part-Time} in \textit{Status}. We also performed some \textbf{data aggregation}, estimating the \textit{Annual Salary} of hourly employees by using the formula \textit{Typical Hours} * \textit{Hourly Rate} * 52, where 52 is a constant representing the number of weeks in a year.

Since our focus is gender pay gap but the original datasets do not contain a \textit{Gender} attribute, we adopted a Python package called \texttt{gender-guesser}\footnote{Available at: \url{https://pypi.org/project/gender-guesser}.}. The aim of the package is to infer a person's gender from their first name, and the possible outcomes are: unknown (name not found), andy (androgynous), male, female, mostly\_male, or mostly\_female. The difference between andy and unknown is that the former is found to have the same probability to be male than to be female, while the latter means that the name was not found in the database. For each employee, we split the \textit{Name} attribute to obtain their \textit{First Name}, and then we inferred their gender by using the package. We obtained (out of the total of 31,858 tuples):
\begin{itemize}
\item unknown: 2653 values.
\item andy: 184 values.
\item male: 20,562 values.
\item female: 6,954 values.
\item mostly\_male: 775 values.
\item mostly\_female: 730 values.
\end{itemize}
In order to get coherent results in case of multiple experiments on the same dataset, we decided to remove the tuples related to unknown and androgynous names instead of randomly assign a gender to them (otherwise, we would have get different numbers at each execution of the preprocessing algorithm). Furthermore, we assumed mostly male names to be effectively related to males and mostly female names to be effectively related to females, and therefore we got 21,337 male values and 7,684 female values for a newly generated \textit{Gender} attribute as a result of this first \textbf{data cleaning} process.

We also operated \textbf{data reduction} by removing the \textit{Typical Hours}, \textit{Hourly Rate} and \textit{First Name} columns, not relevant for our analysis.

As a consequence of the first data cleaning process, the number of different job titles decreased from 1,089 to 1,057. However, since the FAIR-DB tool used for bias analysis requires user interactions, and in order to lighten the workload and speed up computational times, we decided to remove job titles with less than 100 occurrences.

Our final preprocessed dataset includes 20,309 tuples, of which 16,146 males and 4,163 females, and with 35 distinct \textit{Job Title} values.


\newpage
\subsection{The ``Glassdoor Method''}
\subsection{FAIR-DB}
\subsection{Ranking Facts}


\section{Case Study 2: San Francisco}
\subsection{Data Preprocessing}
\subsection{The ``Glassdoor Method''}
\subsection{FAIR-DB}
\subsection{Ranking Facts}


\section{Other Design Choices}


\iffalse

\note{Figures and tables}{You are an engineer, and using figures (illustrations) and tables to better convey your ideas should be an obvious practice you should have learned throughout your university career. If not, it's time now. Use illustrations, screen shots, sketches, and so on to help the reader understand. Use tables to summarize complex text (for example, a profound analysis of the state of the art) or to format data in a readable fashion. Each time you use a figure or table, you must also (i) complement it with a so-called caption (a text right underneath or above it) to give it a title and a description and (ii) reference it from within the main text (never just place a figure somewhere without talking about it). If you use Latex, check your Latex documentation for how to use captions and references.}

\fi
