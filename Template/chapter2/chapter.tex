% !TEX root = ../thesis.tex
\chapter{State of the Art}
\label{capitolo2}
\thispagestyle{empty}

This chapter discusses the state of the art that is relevant for your own work. What does that mean? It means that it provides the reader with all the relevant references he/she may need to know in order to understand better three things: (i) the context of your work, (ii) the problem and the need for a solution, and (iii) the value of your contribution. You achieve this by citing works or scientific papers that solved the same or similar problems in the past. Citing does not just mean adding a references to the bibliography and printing a number here; it means you tell the reader about the merits and possible demerits of each of the references you feel relevant. Of course, doing so requires you to first read each reference and, most importantly, to understand it. There should be lots of references in this chapter. 

It is advisable that you structure the chapter into sections in function of the topics you treat. If you do so, before starting with the first section of the chapter, explain the reader how you structure your discussion in one paragraph.

\begin{itemize}
\item[\Square] \emph{Read} relevant literature and or \emph{test} related software or tools.
\item[\Square] \emph{Summarize} your reading.
\item[\Square] Provide correct \emph{references} (the bibliography in the end of this document).
\end{itemize}


\section{Bias}
Although the word ``\textbf{bias}'' doesn't have an intrinsically negative meaning, it is mostly used in contexts where it entails a moral dimension. As reported in \cite{friedman2017bias}, ``we use the term bias to refer to computer systems that \textit{systematically} and \textit{unfairly discriminate} against certain individuals or groups of individuals in favor of others. A system discriminates unfairly if it denies an opportunity or a good or if it assigns an undesirable outcome to an individual or group of individuals on grounds that are unreasonable or inappropriate''. Therefore, it is important to underline that unfair discrimination due to bias is strictly related to systematic and unfair outcome.

By following the classification provided in \cite{friedman2017bias}, we can distinguish three overarching categories of bias:
\begin{itemize}
\item \textbf{Preexisting Bias}: it has its roots in social institutions, practices and attitudes. Preexisting bias may originate in the society at large, in subcultures and organizations, and can enter a computer system either voluntarily or implicitly and unconsciously, even in spite of the best intentions of the system designer.
\item \textbf{Technical Bias}: it arises from the resolution of issues in the technical design. Technical bias may originate from design choices, constraints and technological tools.
\item \textbf{Emerging Bias}: it emerges some time after a design is completed, as a result of changing societal knowledge, population, or cultural values. Emerging bias is strictly related to a context of use, and it is the most difficult to detect.
\end{itemize}

For the purpose of this research, we will focus on preexisting bias (in particular, \textit{societal bias}) and technical bias, but it is important to point out that emerging bias shouldn't be underestimated in the long run, especially when it arises from a \textit{mismatch between users and system design} due to \textit{different values}, that is bias originated when a computer system is used by a population with different values than those assumed in the design, because society is in constant change and systems should be readjusted or reinvented in order to keep up with the present.

\section{[Topic two]}
...

\section{Summary}
Close the state of the art chapter with some words that connect the discussion of the references to your thesis. Pay attention that the reader understands why you discussed the works/topics you discussed and how they are related to what you do.

\begin{itemize}
\item[\Square] Show that in the state of the art the \emph{problem} you want to solve has not yet been solved or not been solved in an as efficient / effective / easy to use / cost-saving fashion as you target with your work.
\item[\Square] If your work has similarities with some \emph{specific references}, point them out here and explain why these are particularly important to you. Perhaps you started your investigation from the outputs of a specific paper or you want to improve the performance of an algorithm studied earlier; it's good to mention this here.
\item[\Square] Attention: this is not yet the place where to anticipate \emph{your solution}. You may give hints, but it's too early to make a comparison between your work and the state of the art, as the reader does not yet know anything about your work. This discussion can go into the final chapter.
\end{itemize}
