% !TEX root = thesis.tex

\newpage
\chapter*{Abstract}

\addcontentsline{toc}{chapter}{Abstract}

The society in which we live today, like every other in human history, is plagued by various social problems. One of the main differences from the past, however, is linked to the pervasive use of technology, which characterizes and very often facilitates our daily lives, but which, in the context of social problems, can lead to the introduction or the exacerbation of the problems themselves, even at the expense of the best intentions of the user.

Because of their intrinsic complexity, these issues require to be addressed from different but complementary perspectives, and this thesis work would therefore like to be a bridge between two disciplines of very different nature, which at first sight would seem to have little to do with each other: \textit{data science} and \textit{sociology}. As for the former, our focus is on the area of \textit{data analysis}, while for what concerns the latter, we focus on a specific category of social problems, namely that of discrimination, and in particular \textit{gender discrimination}.

To do so, we use an approach that has data analysis as its starting point, and that finds in sociology a useful supporting instrument, as well as a source of requirements. We investigate in depth the sociological reasons behind gender discrimination in the reference society -- the American one -- introducing and exploring what is commonly referred as `\textbf{gender gap}', and we carry out several experiments on data related to U.S. employees, focusing on the economic perspective (\textit{gender pay gap}) but taking into account the different other facets of the problem.

The main contributions of this thesis derive from the application of preprocessing techniques and from the use of three tools created with the aim of detecting \textbf{bias} in the data, with which we try to understand which design choices have the greatest impact on the so-called `\textbf{fairness}' of the results, and of which we highlight strengths and weaknesses, emphasizing the importance of a multidisciplinary approach to problems of this kind, that is essential to obtain information on the very complex context in which data are embedded.

This thesis work was carried out in France at the \'Ecole normale sup\'erieure in Paris, under the supervision of Pierre Senellart, Professor of Computer Science at the aforementioned university, and Karine Gentelet, Associate Professor in the Department of Social Sciences at the Universit\'e du Qu\'ebec en Outaouais (Canada) and holder of the 2020-2021 Research Chair in AI and Social Justice at the \'Ecole normale sup\'erieure.


\newpage
\chapter*{Sommario}

\addcontentsline{toc}{chapter}{Sommario}

La societ\`a in cui viviamo oggigiorno, cos\`i come ogni altra nella storia dell'uomo, \`e afflitta da diverse problematiche sociali. Una delle principali differenze rispetto al passato, per\`o, \`e legata all'utilizzo pervasivo della tecnologia, che caratterizza e molto spesso facilita la nostra vita quotidiana, ma che, nel contesto dei problemi sociali, pu\`o portare all'introduzione o all'esacerbazione degli stessi, anche a discapito delle migliori intenzioni dell'utilizzatore.

A causa della loro complessit\`a intrinseca, questi problemi devono essere affrontati da prospettive differenti ma tra loro complementari, e questo lavoro di tesi vorrebbe quindi essere un ponte tra due discipline di natura molto diversa, che all'apparenza sembrerebbero avere poco a che fare l'una con l'altra: \textit{data science} e \textit{sociologia}. Per quanto concerne la prima, il nostro focus \`e sull'ambito dell'\textit{analisi dei dati}, mentre per quanto riguarda la seconda, ci concentriamo su una specifica categoria di problemi sociali, ovvero quella della discriminazione, e in particolare della \textit{discriminazione di genere}.

Per farlo, utilizziamo un approccio che ha come punto di partenza l'analisi dei dati, e che trova nella sociologia un utile strumento di supporto, nonch\'e fonte di requisiti. Indaghiamo a fondo sulle ragioni sociologiche alla base della discriminazione di genere nella societ\`a di riferimento -- quella americana -- introducendo ed esplorando quello che viene comunemente definito `\textbf{gender gap}', ed effettuiamo diversi esperimenti su dati relativi a impiegati statunitensi, concentrandoci sulla dimensione economica (\textit{gender pay gap}) ma tenendo in considerazione le diverse altre sfaccettature del problema.

I principali contributi di questa tesi derivano dall'applicazione di tecniche di data preprocessing e dall'utilizzo di tre strumenti creati con l'obiettivo di rilevare \textbf{bias} nei dati, con i quali cerchiamo di capire quali scelte progettuali impattino maggiormente sulla cosiddetta `\textbf{fairness}' dei risultati, e dei quali mettiamo in luce punti di forza e di debolezza, sottolineando l'importanza di un approccio multidisciplinare a problemi di questo tipo, indispensabile per ottenere informazioni sul contesto molto complesso in cui i dati sono inseriti.

Questo lavoro di tesi \`e stato svolto in Francia presso l'\'Ecole normale sup\'erieure di Parigi, sotto la supervisione di Pierre Senellart, professore di Computer Science nella suddetta universit\`a, e Karine Gentelet, professoressa associata nel dipartimento di Social Sciences presso l'Universit\'e du Qu\'ebec en Outaouais (Canada) e detentrice della cattedra di ricerca 2020-2021 in AI and Social Justice all'\'Ecole normale sup\'erieure.
