% !TEX root = thesis.tex

\newpage
\chapter*{Abstract}

\addcontentsline{toc}{chapter}{Abstract}

The society in which we live today, like every other in human history, is plagued by various social problems. One of the main differences from the past, however, is linked to the pervasive use of technology, which characterizes and very often facilitates our daily lives, but which, in the context of social problems, can lead to the introduction or the exacerbation of the problems themselves, even at the expense of the best intentions of the user.

This thesis work would therefore like to be a bridge between two disciplines of very different nature, and which at first sight would seem to have little to do with each other: \textit{data science} and \textit{sociology}. As for the former, our focus will be on the area of \textit{data analysis}, while for what concerns the latter, we will focus on a specific category of social problems, namely that of discrimination, and in particular \textit{gender discrimination}.

To do so, we will investigate in depth the sociological reasons behind gender discrimination in the reference society -- the American one -- introducing and exploring what is commonly referred as `\textbf{gender gap}', and we will carry out several experiments on data related to U.S. employees, focusing on the economic perspective (\textit{gender pay gap}) but taking into account the different other facets of the problem.

Through preprocessing techniques and the use of three tools created with the aim of detecting \textbf{bias} in the data, we will then try to understand which design choices impact more on the so-called `\textbf{fairness}' of results, and we will highlight the strengths and weaknesses of these tools, underlining the importance of a multidisciplinary approach to problems of this kind, which is essential to obtain information on the context in which data are embedded.


\newpage
\chapter*{Sommario}

\addcontentsline{toc}{chapter}{Sommario}

La societ\`a in cui viviamo oggigiorno, cos\`i come ogni altra nella storia dell'uomo, \`e afflitta da diverse problematiche sociali. Una delle principali differenze rispetto al passato, per\`o, \`e legata all'utilizzo pervasivo della tecnologia, che caratterizza e molto spesso facilita la nostra vita quotidiana, ma che, nel contesto dei problemi sociali, pu\`o portare all'introduzione o all'esacerbazione degli stessi, anche a discapito delle migliori intenzioni dell'utilizzatore.

Questo lavoro di tesi vorrebbe quindi essere un ponte tra due discipline di natura molto diversa, e che all'apparenza sembrerebbero avere poco a che fare l'una con l'altra: \textit{data science} e \textit{sociologia}. Per quanto concerne la prima, il nostro focus sar\`a sull'ambito dell'\textit{analisi dei dati}, mentre per quanto riguarda la seconda, ci concentreremo su una specifica categoria di problemi sociali, ovvero quella della discriminazione, e in particolare della \textit{discriminazione di genere}.

Per farlo, indagheremo a fondo sulle ragioni sociologiche alla base della discriminazione di genere nella societ\`a di riferimento -- quella americana -- introducendo ed esplorando quello che viene comunemente definito `\textbf{gender gap}', ed effettueremo diversi esperimenti su dati relativi a impiegati statunitensi, concentrandoci sulla dimensione economica (\textit{gender pay gap}) ma tenendo in considerazione le diverse altre sfaccettature del problema.

Attraverso tecniche di data preprocessing e l'utilizzo di tre strumenti creati con l'obiettivo di rilevare \textbf{bias} nei dati, cercheremo quindi di capire quali scelte progettuali impattino maggiormente sulla cosiddetta `\textbf{fairness}' dei risultati, e metteremo in luce punti di forza e di debolezza di questi strumenti, sottolineando l'importanza di un approccio multidisciplinare a problemi di questo tipo, indispensabile per ottenere informazioni sul contesto nel quale i dati sono inseriti.
