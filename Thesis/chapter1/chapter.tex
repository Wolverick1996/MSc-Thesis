% !TEX root = ../thesis.tex
\chapter{Introduction}
\label{chapter:introduction}
\thispagestyle{empty}


\section{Research Context: Gender Discrimination in Data Analysis}
This research would like to be a bridge between two disciplines of very different nature, and which at first sight would seem to have little to do with each other: \textit{data science} and \textit{sociology}. As for the former, our focus will be on \textit{data analysis}, that is, the set of processes for inspecting, cleaning, transforming, and modeling data with the aim of discovering useful information, informing conclusions, and supporting decision making. For what concerns the latter, we will focus on \textit{social justice}, and even though most of the definitions will be provided in further chapters, it is appropriate to clarify the very 
fundamental concepts behind our research, starting from the notion of ``social problem''.
\begin{quote}\emph{Social problem is a generic term applied to a range of conditions and aberrant behaviors which are manifestations of social disorganization. It is a condition which most people in a society consider undesirable and want to correct by changing through some means of social engineering or social planning.} \cite{marschall1998oxford}\end{quote}

In our research, we will focus on a specific category of social problems, namely those related to discrimination, and in particular on \textit{gender discrimination}, expressed in the form of the so-called ``\textbf{gender gap}''. According to \cite{cambridge2013gender}, gender gap is definable as:
\begin{quote}\emph{A difference between the way men and women are treated in society, or between what men and women do and achieve.} \cite{cambridge2013gender}\end{quote}

Specifically, the focus of our experiments will be \textit{gender pay gap}, that is, the average difference between the remuneration for men and women in the workforce, or, in other words, a measure of what women are paid relative to men. Our experiments will be centered on the economical perspective because it is the easiest to be measured in the data, being quantifiable for example as a number representative of a person's average monthly salary, but it is important to keep in mind that other facets of the gender gap problem come into play when dealing with sociological studies.


\section{Scenarios \& Problem Statement}
Nowadays, digital devices have become pervasive in every aspect of our daily lives and almost every action we perform leaves a digital trail. We generate data whenever we go online, when we communicate with people through any kind of application, when we shop, or even just when we carry our smartphones. It is therefore of societal and ethical importance to ask whether data and datasets, on which so many actions of our daily routine are based, are \textit{fair} or not. Unfair, or better to say, \textit{biased} data, may in fact influence, directly or indirectly, our perception of reality, and lead us to make decisions that, although seemingly fair and just, contain in turn bias, and therefore discriminate against individuals or groups of individuals. Without going into too much detail, which will be deepened in the following chapters, we will now provide the reader with a couple of scenarios that can intuitively give the idea of the problem.

A first example, related to racial discrimination, is given by \cite{ledford2019millions}. A study conducted by the University of California in 2019 concluded that an algorithm used to allocate health care to patients in U.S. hospitals was less likely to refer black people than white people who were equally sick to programs that aim to improve care for patients with complex medical needs. Although there was no discriminatory intent, misconceptions in the design phase led to the introduction of bias, and consequently to the involuntary discrimination of millions of black citizens.

Another example, more closely linked to the focus of our research, namely gender discrimination, is provided in \cite{horwitz2021facebook}. The article reports the outcomes of a study led by the University of Southern California researchers in 2021, who found that Facebook systems were more likely to present job ads to users if their gender identity mirrored the concentration of that gender in a particular position or industry. Specifically, the team of researchers bought ads on Facebook for delivery driver job listings that had similar qualification requirements but for different companies: Domino's, who has more male drivers, and Instacart, who has more female ones. The study found that Facebook targeted the Instacart delivery job to more women and the Domino's delivery job to more men. Similar findings were obtained by testing software engineer job listings for Nvidia and Netflix, and therefore the researchers, in their conclusions, spoke of ``a platform whose algorithm learns and perpetuates the existing difference in employee demographics''.

Further examples and explanations will be provided in the following chapters, but what we want to clarify is that our goal is not to solve the problem of discrimination in data, which is a huge and multifaceted issue which encompasses a human dimension very difficult to address, but rather to take a look at the current state of the art, observing some tools in action and trying to highlight their strengths and weaknesses, and also providing a non-technical perspective to give a broader picture of the situation. We believe that this research may represent a good starting point for those who work in the sector to have an overview of the problem and to better understand what to focus on.


\section{Methodology}
We will approach the problem by first providing the reader with some preliminary socio-ethical and technical concepts taken from a \textit{systematic literature review}, and later on we will describe, referring to the documentation at our disposal, the tools we decided to adopt for our analysis.

After that, we will conduct \textit{parallel research} in sociology and information technology, distinguishing some \textit{case studies} for our experiments and trying to interpret our results also looking at the sociological background introduced beforehand.
The experiments themselves are basically \textit{comparative studies} of the algorithms, and, in terms of \textit{software instruments}, we will rely mainly on Python and Jupyter Notebook, with the various packages and libraries made available by developers.


\section{Thesis Structure}
We provide here a brief overview of the contents of each chapter, in order to help the reader move through the thesis and find specific information. The beginning of each chapter provides more detailed descriptions about the contents of the chapter itself.

\begin{itemize}
\item Chapter 2: \textbf{Socio-Ethical Preliminaries}. The aim of this chapter is to provide the reader with preliminary notions of ethical and sociological (rather than technical) nature.
\item Chapter 3: \textbf{Technical Preliminaries}. The aim of this chapter is to provide the reader with preliminary notions about the technical knowledge necessary to understand the functioning of the adopted tools, which will be described later on in Chapter~\ref{chapter:techniques}.
\item Chapter 4: \textbf{Sociological Research}. The aim of this chapter is to provide the reader with a sociological background by reporting information about gender gap in the society.
\item Chapter 5: \textbf{Techniques}. The aim of this chapter is to provide the reader with an overview on the tools adopted in our experiments.
\item Chapter 6: \textbf{Experiments}. The aim of this chapter is to describe the experiments conducted using the tools introduced in Chapter~\ref{chapter:techniques}, in order to verify the presence (and the nature) of bias in our datasets and discuss their fairness, according to the concepts introduced in Chapter~\ref{chapter:socio-ethical_preliminaries} and the sociological background explored in Chapter~\ref{chapter:sociological_research}.
\item Chapter 7: \textbf{Conclusions \& Future Work}. The aim of this chapter is to draw the conclusions of our research, mixing the results obtained in our experiments described in Chapter~\ref{chapter:experiments} with the sociological background depicted in Chapter~\ref{chapter:sociological_research}, and recalling some preliminaries exposed in Chapter~\ref{chapter:socio-ethical_preliminaries} and Chapter~\ref{chapter:technical_preliminaries} when needed.
\end{itemize}
